\documentclass[a4paper]{article}

\usepackage{hyperref}
\hypersetup{
colorlinks=false, % bool: Liens colorés
pdfborder={0 0 0} % Ne pas encadrer les liens
}
\usepackage[utf8]{inputenc}
\usepackage[francais]{babel}
\usepackage[top=2cm, bottom=2cm, left=2cm, right=2cm]{geometry}
\usepackage{graphicx}
\usepackage[final]{pdfpages}
\usepackage{rotating}
\usepackage{eurosym}
\usepackage{lscape}
\usepackage{float}
\usepackage{color}
\usepackage{colortbl}
% définir les commandes ici
\newcommand{\titlecolor}[1]{\textcolor{blue}{\section{#1}}}

% s'il y a beaucoup de commandes et de packages à inclure n'h&ésitez pas
% à mettre tout ça dans un fichier include.tex et l'inclure
% \input{include.tex}




\begin{document}
\title{Synthèse bibliographique : Nettoyage des données en présence de données de références}

\author{Elisa ABIDH, Julien LEVESY, Armand ROSSIUS}

%------------------------------------- Page de titre
\maketitle
%\begin{titlepage}
%~

%\vfill
%\begin{Large}
%Septembre 2011
%\end{Large}
%\vfill
%\end{titlepage}
%----------------------------------------------------

%--------------------------------- Table des matières
\newpage
\tableofcontents
\newpage
%----------------------------------------------- Plan

\subsection {Enjeux de la qualité de données}
Depuis plusieurs années la gestion des données est devenue cruciale pour les entreprises, le volume de données stockées et échangées augmentent, ce qui confronte les entreprise à la problématique de comment stocker les données et pouvoir y accéder facilement, on parle alors de  "Data Management" ou gestion des données.\\ Que ce soit pour des raisons légales, des besoins opérationnelles ou pour des choix stratégiques la gestion de l'information est importante.  Au sein d'une entreprise beaucoup d'activités et fonctions sont concernées :
\begin{itemize}
\item[-] La gestion d’activité optimale pour répondre à la demande : demande une maitrise de l’information.
\item[-] Toutes les fonctions des entreprises sont gérées par le SI
\item[-] Les données sont un flux présent dans toutes les entreprises
\item[-] Les dirigeants : parce que les décisions, le plan stratégique nécessite de l’information
\item[-] Les responsables opérationnels : ils traitent de l’information pour pouvoir gérer au mieux les problèmes. 
\item[-] Marketing : données sur les fournisseurs, les clients, les concurrents, les marchés
\item[-] Les collaborateurs opérationnels : approvisionner le stock, lister des interventions sur une machine, nom des pièces changées
\end{itemize}
On comprend que la qualité des données gérer par le SI est importante pour répondre aux attentes du client mais aussi pour gérer de manière optimale l'entreprise. Cependant l'entreprise va rencontrer plusieurs difficultés pour répondre à cette problématique de  qualité de données : 
\begin{itemize}
\item[-]Détecter la mauvaise qualité des données.
\item[-]Trop de données car beaucoup de données inutiles. 
\end{itemize}
Avant de répondre à cette problématique il est légitime de se poser la question : "Qu'est ce qu'une donnée de qualité ? "
D'après le livre de Christophe Brasseur \cite{Brasseur} la qualité ne se résume pas à une donnée juste, c'est une condition nécessaire mais non suffisante. Il est difficile de donner une définition précise de cette notion, cependant on peut dégager plusieurs axes pour juger de la qualité des données : qualité du contenu, accessibilité, flexibilité, sécurité.
\begin{enumerate}
\item {Qualité du contenue}
\begin{itemize}
\item[-]Justesse de l’information : en phase avec la réalité
\item[-]Adéquation aux besoins : réponds aux besoins réels
\item[-]Facilité d'interprétation : pas d'ambiguïté(abrévation,unités), compréhensible.
\end{itemize}
\item {Accessibilité}
\begin{itemize}
\item[-]Disponibilité : disponible quand on en a besoin
\item[-]Facilité d’accès : ergonomie des applications.
\end{itemize}
\item {Flexibilité}
\begin{itemize}
\item[-]Evolutivité : définition et codification de la donnée (pas de remise en cause) 
\item[-]Cohérence avec d’autres sources (identifier les données partagées), 
\item[-]Possibilité de traduction.
\end{itemize}
\item {Sécurité}
Protéger l’information des menaces accidentelles et des attaques malveillantes
\begin{itemize}
\item[-]Confidentialité 
\item[-]Fiabilité 
\item[-]Traçabilité
\item[-]Intégrité des données.
\end{itemize}
\end{enumerate}



\section{Une réponse : le master data management}

\subsection{Présentation du procédé et objectifs} 

Le Master Data Management, traduit en français par Gestion des données de références, est une discipline des technologies de l'information ayant pour objectif de définir des concepts et méthodes visant à établir au sein d'un système d'information un schéma de base de données de références considérées commes fiables.\\
Outre cela, le Master Data Management engloble aussi les disciplines d'intégration, d'exposition et d'utilisation de ces données de références au sein d'un système d'information d'entreprise, autant du coté opérationel que analytique.\\
Ce procédé, permet de répondre en partie à la problématique de la qualité des données, en définissant un cadre de données dites de références, sures, et limite ainsi l'entropie des données intégrées au DataWarehouse, mais n'effectue pas à proprement parler de nettoyage des données, thème qui sera abordé dans la suite de la synthèse\\

\subsection{Principes}

L'hypothèse de base est la suivante : \textit{"En assurant la qualité sur les données de références, on limite les erreurs lors de l'alimentation et l'exploitation de l'entrepot de données"}\\

\subsubsection{Données de références, késako ? }

Les données de références sont un sous enssemble des données opérationelles, qui ont la praticularité de ne pas êtres issues d'opération de transactions. Ainsi elle possèdent une certaine constance dans le temps, qui n'est cependant pas une invariance, ces données pouvant être modifiées, complétées voire étendues. Ce sont ces mêmes données qui vont définir les axes d'exploration, d'exploitation et d'analyse.\\
On différencie trois grandes catégories de données de références.

\begin{itemize}
\item Produit : Chaque entreprise possède une quantité de réféence produits, qui peuvent êtres transersaux à plusieurs secteurs de l'entreprise. Typiquement, un produit pourra être référencé par une documentation technique issue d'un bureau d'étude, une opération de vente  ou encore un référenciel fournisseur. L'unicité devra donc être assurée sur l'enssemble des entrées dans ce domaine.
\item Tiers : De façon similaires, les "tiers" d'entreprises sont aussi considérés comme données de références. Par tiers nous entendons toutes personne ou entité ayant une intéraction possible avec le système d'information, typiquement un collaborateur, un client ou encore un fournisseur.
\item Finance  : Les données de finances sont des informations critiques pour le fonctionnement de l'entreprise, obligatoire en ce qui concerne n'importe quel aspect légal et primoridal en ce qui concerne le pilotage des activités. Ces deux approches sont intégrées aux données de références.
\end{itemize}


\subsubsection{Pourquoi définir un parc de données de références ?}

\subsubsection{Positionnement au sein du SI de l'entreprise}

\subsubsection{Stratégie d}

\subsection{Implémentation dans un système d'information d'entreprise}

\subsection{Présentation des offres du marché}

\subsubsection{Oracle}

\subsubsection{IBM}





\end{document}
