\documentclass[a4paper]{article}

\usepackage{hyperref}
\hypersetup{
colorlinks=false, % bool: Liens colorés
pdfborder={0 0 0} % Ne pas encadrer les liens
}
\usepackage[utf8]{inputenc}
\usepackage[francais]{babel}
\usepackage[top=2cm, bottom=2cm, left=2cm, right=2cm]{geometry}
\usepackage{graphicx}
\usepackage[final]{pdfpages}
\usepackage{rotating}
\usepackage{eurosym}
\usepackage{lscape}
\usepackage{float}
\usepackage{color}
\usepackage{colortbl}
% définir les commandes ici
\newcommand{\titlecolor}[1]{\textcolor{blue}{\section{#1}}}

% s'il y a beaucoup de commandes et de packages à inclure n'h&ésitez pas
% à mettre tout ça dans un fichier include.tex et l'inclure
% \input{include.tex}




\begin{document}
\title{Synthèse bibliographique : Nettoyage des données en présence de données de références}

\author{Elisa ABIDH, Julien LEVESY, Armand ROSSIUS}

%------------------------------------- Page de titre
\maketitle
%\begin{titlepage}
%~

%\vfill
%\begin{Large}
%Septembre 2011
%\end{Large}
%\vfill
%\end{titlepage}
%----------------------------------------------------

%--------------------------------- Table des matières
\newpage
\tableofcontents
\newpage
%----------------------------------------------- Plan

%\input{.tex}




\end{document}
