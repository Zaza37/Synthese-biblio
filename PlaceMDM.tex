\section{Positionnement au sein du SI de l'entreprise}

\subsection{ Un peu d'histoire...}

Historiquement, à l'âge (pas tant) de pierre (que ça) du système d'information, chaque application opérationnelle possédait son propre SGBD dédié à l'application... Celle-ci ne possédait que les données qui lui étaient utiles, que ce soit de références, ou de simples transactions.\\
Le problème de la propagation des mises à jour des données est alors posé, car laissé à la responsabilité de l'opérateur, et comme le dit l'adage \textit{" La seule source d'erreur possible dans un ordinateur se trouve entre la chaise et le clavier !"}.\\
La continuité logique des choses est donc d'essayer de "faire communiquer" les différents SGBDs entre eux... Vient donc la problématique de l'intégration n-carrée : chaque application est raccordée directement aux multiples bases de données qu'elle utilise, sans réel moyen de contrôle de la mise à jour de ces dernières... Fort risque de corruption lors de la propagation de données, de création de doublons sur certaines entrées et aucune trace des modifications portées. \\
Ce système s'est donc révélé catastrophique en terme de maintenance et de qualité des données, mais il avait au moins le mérite d'avoir permis d'identifier une solution possible à la propagation des données au sein d'un SI: il faut contrôler et uniformiser les modes de communication entre les différentes bases de données\\

\subsection{ EAI : Intégration d'application opérationnelle dans le SI d'entreprise}

Compte-tenu des expériences décrites précédemment, les développeurs ont orienté la démarche vers la création d'un bus commun de communication entre les différentes  entités du système d'information. Ainsi ce service fourni sera en charge de l'archivage et du transit des données de l'entreprise le tout de façon générique, moyennant le développement de services "connecteurs" entre les applications et le système de communication, appelé Entreprise Service Bus, ou ESB.\\

\begin{itemize}

\item Applications Opérationnelles : Applications métier de l'entreprise, raccordées à l'ESB, 

\item Synchronisation des données basées sur les méta données. Toutes les informations traitant  des opérations à effectuer sur les différentes bases sont stockées à part. Ainsi la tâche de synchronisation du contenu est externalisée. Cela est aussi appelé ESB.

\item Les fonctionnalités clés des applications métiers sont maintenant exposées  comme "services" dans un système d'orchestration. Ainsi, il est possible de définir plusieurs processus d'orchestration de services. En ce qui concerne le MDM, il ne s'agit pas de seulement être en A2A (Application to application), mais aussi d'exposer les données de références à la couche d'orchestration.

\end{itemize}

\subsection{Limites du modèle EAI \& SOA}

Les architectures de gestion de systèmes d'informations présentées précédemment présentent l'avantage de faciliter l'intégration de nouveaux services à un système d'information d'entreprise, en s'occupant principalement de la problématique de communication / synchronisation des données entre plusieurs applications opérationnelles. En d'autres termes, ils sont conçus pour gérer et limiter les problèmes de fragmentation, mais ils ne les éliminent pas.\\
Il convient alors de déployer une solution, autre que du nettoyage instanciel bête et méchant, en adoptant une approche plus intelligente, basée sur une vision au long terme. 

\section{Présentation des offres du marché}

Pour appuyer mon analyse, je me suis basé sur les études menées par un cabinet de consultant, intitulé the MDM Institute.

\subsection{Oracle}

La solution oracle tire parti de l'expérience de l'éditeur logiciel en terme d'outils pour l'intégration d'entreprise. En effet, ce dernier possède déjà un outil de service bus, d'entreposage des données et une multitude d'outils dédiés à l'intégration d'applications opérationnelles d'entreprise.\\
Jouissant de la maitrise de ces couches fondamentales du SI et nécessaire  au déploiement d'un système de MDM performant, l'éditeur est donc en position de force pour fournir une solution efficace... avec sa "stack" d'intégration seulement ! \\
L'éditeur pris le parti de d'approcher le domaine en segmentant ce dernier en domaines spécifiques, basés sur la toute relative catégorisation des données présentée un peu plus haut dans cette synthèse.\\
On retrouve une offre logicielle par type de données de référence, Oracle prenant le parti de dire qu'en différenciant l'offre logicielle par domaine spécifique des données de référence, ils seront plus facilement apte à fournir des services de gestion des données de références plus ciblés et donc plus efficaces.\\
Les grandes étapes de ce traitement sont donc :\\

\begin{itemize}
\item Govern
\item Consolidate
\item Cleanse
\item Share
\end{itemize}

Reste à savoir comment gérer les données plus transversales, le prédicat de base donnant répondant à la définition de la donnée de référence par une réponse simpliste...\\

\subsubsection{Tiers}

Premier domaine des données de références, les données liées aux "tiers" de l'entreprise. Ici Oracle à  choisir de fournir une solution dédiée à la gestion des données client, différenciée des fournisseurs, ce qui est compréhensible les données client, sont différentes des fournisseurs et la collision de ces deux catégories d'entreprises est rarement possible.\\

L'objectif de cette solution est de permettre la création d'une seule et unique vue d'un client ou du fournisseur en exploitant des sources hétérogènes au sein du SI, typiquement une application de ressources client, une application opérationnelle, de la gestion de fournisseurs, etc...\\

\begin{itemize}
\item Oracle Customer Hub
\item Oracle Supplier Hub
\end{itemize}

\subsubsection{Produit}

La gestion des références produit est un problème important au sein d'entreprises techniques, en partie lié au fait que c'est un domaine extrêmement transversal ( Ventes, Ingéniérie, Logistique ...) dans le contexte de l'entreprise. L'objectif la encore est de fournir une vue unique du produit en cours, peu importe le domaine dont est issu l'information.\\

\begin{itemize}
\item Oracle Product Hub
\end{itemize}

\subsubsection{Analytique}

Troisième domaine, les données analytiques et financières. Ces deux domaines, extrêmements liés, représentent un challenge important au sein des solutions de master data management par le fait qu'elle ne sont pas dépendantes d'un modèle figé. Ainsi, concevoir une base de données de référence sur un modèle de données fortement instable est un problème extrêmement compliqué.\\
L'éditeur fais ici le choix d'aborder le problème sous l'angle de la relation entre les données de références, d'où le nom du logiciel.\\

\begin{itemize}
\item  Oracle Hyperion Data Relationship Management
\end{itemize}

\subsubsection{Critique de la solution}

La force de cette solution comparé à la concurrence réside évidemment dans la qualité des traitement et des modèles fournis, cependant les domaines sont trop orientés et le potentiel d'évolution des fonctionnalités, en parti bridé par le prédicat de la séparation des données de références, est vivement critiqué.\\
Cette approche est efficace, mais pas viable sur le long terme\\ 
Enfin, notons la relative fermeture de ce système, seules les solutions oracles sont intégrables, aucune hypothèse de SI hétérogène n'est envisageable.

\subsection{IBM}

A l'inverse d'Oracle, IBM à une approche plus globale, plus générique du Master Data Management.\\
Ainsi, le support de nombreux domaines opérationnels de l'entreprise, des nombreux cas d'utilisations sont centralisés au sein d'un seul et même service.\\
La encore, la solution de MDM proposée par IBM jouis d'une très forte synergie avec les autres outils de systèmes d'information fournis par l'éditeur, tels que les utilitaires d'orchestration de processus SOA, ou encore de data center.\\
L'approche centralisée de cette solution permet de renforcer la cohérence du modèle MDM sur la totalité des activités gérées par ce dernier et de minimiser les problèmes d'interopérabilité entre les différents domaines de données de références que nous pourrions rencontrer chez Oracle par exemple.\\
De plus, l'intégration au sein d'un Si existant se révelle facilitée, la solution étant plus facilement intégrable du fait de son approche générique et centralisée, et beaucoup plus ouvertes en terme d'interopérabilité avec d'autres utilitaires de gestion de données issus d'autres éditeurs logiciels.\\
Enfin, la généricité des traitements appliqués aux données de référencent offrent  une plus grande flexibilité en ce qui concerne les cas d'utilisation, offrant une vision beaucoup plus durables en termes d'évolution que oracle.\\
Cependant, l'approche générique est beaucoup plus complexes à mettre en œuvre, et certaines technologies de MDM implémentées sont encore susceptibles d'évoluer.\\

\subsection{One Data MDM}

La où les solutions précédentes sont prévues pour être intégrées à une stack logicielle du même éditeur, déjà présente dans le système d'information cible. Il est intéressant  de regarder comment cela se passe pour un outil que nous qualifierons de tiers. \\
One Data est une solution interessante, car elle déploie sa propre plateforme de fonctionnement, de façon totalement indépendante du SI existant.\\
Pour être efficace, ce type de plateforme doit intégrer une solution de centralisation de donnée et d'exposition aux applications opérationnelles.\\
La force de cette suite réside dans sa grande flexibilité accordée au modèle de donnée, en parallèle d'une gestion de templates extrêmement fournie. \\
Cependant, son intégration "plus compliquée" qu'un système MDM cohérent avec le SI pose de nombreuses limitations, notamment au niveau de l'intégration de données.\\
Enfin, son prix reste tout de même extrêmement avantageux !\\
