\section{Démarche}

N'étant pas, par notre formation et nos affinités, des personnes ayant une grande connaissance des problématiques de gestion de données d'entreprise au sein d'un SI, nous avons décidé d'adopter une démarche d'ingénieur "ingénu"  vis à vis du problème posé.\\
Par cela, nous entendons avoir une démarche progressive et critique vis à vis du vaste problème qu'est la qualité des données au sein d'un SI. Ce sujet étant extrêmement vaste et mal défini, nous ne souhaitons pas nous fermer de portes dans notre raisonnement.\\
De plus, nous souhaitons réellement avoir une approche qui permette de distinguer la réalité des intérêts technologiques des arguments commerciaux de bas étage, ces derniers étant plus que mis à contribution  compte tenu du battage médiatique effectué autour de la problématique de la qualité des données. Nous tâcherons de passer outre ces aspects.\\
Ainsi, dans un premier temps, nous allons nous atteler à définir le domaine de la qualité des données, pour ensuite essayer d'effectuer un aperçu des approches possibles de la problématique de qualité des données, pour enfin déboucher sur le management des données de référence, son fonctionnement, son implémentation dans un SI d'entreprise et les différents challenges que ce type de service pose au sein d'une entreprise.\\
De cette manière nous espérons fournir un aperçu complet et critique de cette technologie. 