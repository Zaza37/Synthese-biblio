\section{Une réponse : le master data management}

\subsection{Présentation du procédé et objectifs} 

Le Master Data Management, traduit en français par Gestion des données de références, est une discipline des technologies de l'information ayant pour objectif de définir des concepts et méthodes visant à établir au sein d'un système d'information un schéma de base de données de références considérées commes fiables.\n
Outre cela, le Master Data Management engloble aussi les disciplines d'intégration, d'exposition et d'utilisation de ces données de références au sein d'un système d'information d'entreprise, autant du coté opérationel que analytique.
Ce procédé, permet de répondre en partie à la problématique de la qualité des données, en définissant un cadre de données dites de références, sures, et limite ainsi l'entropie des données intégrées au DataWarehouse, mais n'effectue pas à proprement parler de nettoyage des données, thème qui sera abordé dans la suite de la synthèse

\subsection{Principes}

L'hypothèse de base est la suivante : \textit{"En assurant la qualité sur les données de références, on limite les erreurs lors de l'alimentation et l'exploitation de l'entrepot de données"}

\subsubsection{Données de références, késako ? }

\subsubsection{Pourquoi définir un parc de données de références ?}

\subsubsection{Positionnement au sein du SI de l'entreprise}

\subsubsection{Stratégie d}

\subsection{Implémentation dans un système d'information d'entreprise}

\subsection{Présentation des offres du marché}

\subsubsection{Oracle}

\subsubsection{IBM}
