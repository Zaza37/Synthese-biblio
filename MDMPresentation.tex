\section{Une réponse : le master data management}

\subsection{Présentation du procédé et objectifs} 

Le Master Data Management, traduit en français par Gestion des données de références, est une discipline des technologies de l'information ayant pour objectif de définir des concepts et méthodes visant à établir au sein d'un système d'information un schéma de base de données de références considérées commes fiables.\\
Outre cela, le Master Data Management engloble aussi les disciplines d'intégration, d'exposition et d'utilisation de ces données de références au sein d'un système d'information d'entreprise, autant du coté opérationel que analytique.\\
Ce procédé, permet de répondre en partie à la problématique de la qualité des données, en définissant un cadre de données dites de références, sures, et limite ainsi l'entropie des données intégrées au DataWarehouse, mais n'effectue pas à proprement parler de nettoyage des données, thème qui sera abordé dans la suite de la synthèse\\

\subsection{Principes}

L'hypothèse de base est la suivante : \textit{"En assurant la qualité sur les données de références, on limite les erreurs lors de l'alimentation et l'exploitation de l'entrepot de données"}\\

\subsubsection{Données de références, késako ? }

Les données de références sont un sous enssemble des données opérationelles, qui ont la praticularité de ne pas êtres issues d'opération de transactions. Ainsi elle possèdent une certaine constance dans le temps, qui n'est cependant pas une invariance, ces données pouvant être modifiées, complétées voire étendues. Ce sont ces mêmes données qui vont définir les axes d'exploration, d'exploitation et d'analyse.\\
On différencie trois grandes catégories de données de références.

\begin{itemize}
\item Produit : Chaque entreprise possède une quantité de réféence produits, qui peuvent êtres transersaux à plusieurs secteurs de l'entreprise. Typiquement, un produit pourra être référencé par une documentation technique issue d'un bureau d'étude, une opération de vente  ou encore un référenciel fournisseur. L'unicité devra donc être assurée sur l'enssemble des entrées dans ce domaine.
\item Tiers : De façon similaires, les "tiers" d'entreprises sont aussi considérés comme données de références. Par tiers nous entendons toutes personne ou entité ayant une intéraction possible avec le système d'information, typiquement un collaborateur, un client ou encore un fournisseur.
\item Finance  : Les données de finances sont des informations critiques pour le fonctionnement de l'entreprise, obligatoire en ce qui concerne n'importe quel aspect légal et primoridal en ce qui concerne le pilotage des activités. Ces deux approches sont intégrées aux données de références.
\end{itemize}


\subsubsection{Pourquoi définir un parc de données de références ?}

\subsubsection{Positionnement au sein du SI de l'entreprise}

\subsubsection{Stratégie d}

\subsection{Implémentation dans un système d'information d'entreprise}

\subsection{Présentation des offres du marché}

\subsubsection{Oracle}

\subsubsection{IBM}
