\section {Enjeux de la qualité de données}
Depuis plusieurs années la gestion des données est devenue cruciale pour les entreprises, le volume de données stockées et échangées augmentent, ce qui confronte les entreprise à la problématique de comment stocker les données et pouvoir y accéder facilement, on parle alors de  "Data Management" ou gestion des données.\\ Que ce soit pour des raisons légales, des besoins opérationnelles ou pour des choix stratégiques la gestion de l'information est importante.  Au sein d'une entreprise beaucoup d'activités et fonctions sont concernées :
\begin{itemize}
\item[-] La gestion d’activité optimale pour répondre à la demande : demande une maitrise de l’information.
\item[-] Toutes les fonctions des entreprises sont gérées par le SI
\item[-] Les données sont un flux présent dans toutes les entreprises
\item[-] Les dirigeants : parce que les décisions, le plan stratégique nécessite de l’information
\item[-] Les responsables opérationnels : ils traitent de l’information pour pouvoir gérer au mieux les problèmes. 
\item[-] Marketing : données sur les fournisseurs, les clients, les concurrents, les marchés
\item[-] Les collaborateurs opérationnels : approvisionner le stock, lister des interventions sur une machine, nom des pièces changées
\end{itemize}
On comprends que la qualité des données gérer par le SI est importante pour répondre aux attentes du client mais aussi pour gérer de manière optimale l'entreprise. Cependant l'entreprise va rencontrer plusieurs difficultés pour répondre à cette problématique de  qualité de données : 
\begin{itemize}
\item[-]Détecter la mauvaise qualité des données.
\item[-]Trop de données car beaucoup de données inutiles. 
\end{itemize}
Avant de répondre à cette problématique il est légitime de se poser la question : "Qu'est ce qu'une donnée de qualité ? "
D'après le livre de Christophe Brasseur "Data Management : qualité des données et compétitivité" la qualité ne se résume pas à une donnée juste, c'est une condition nécessaire mais non suffisante. Il est difficile de donner une définition précise de cette notion, cependant on peut dégager plusieurs axes pour juger de la qualité des données : qualité du contenu, accessibilité, flexibilité, sécurité.
\begin{enumerate}
\item {Qualité du contenue}
\begin{itemize}
\item[-]Justesse de l’information : en phase avec la réalité
\item[-]Adéquation aux besoins : réponds aux besoins réels
\item[-]Facilité d'interprétation : pas d'ambiguïté(abrévation,unités), compréhensible.
\end{itemize}
\item {Accessibilité}
\begin{itemize}
\item[-]Disponibilité : disponible quand on en a besoin
\item[-]Facilité d’accès : ergonomie des applications.
\end{itemize}
\item {Flexibilité}
\begin{itemize}
\item[-]Evolutivité : définition et codification de la donnée (pas de remise en cause) 
\item[-]Cohérence avec d’autres sources (identifier les données partagées), 
\item[-]Possibilité de traduction.
\end{itemize}
\item {Sécurité}
Protéger l’information des menaces accidentelles et des attaques malveillantes
\begin{itemize}
\item[-]Confidentialité 
\item[-]Fiabilité 
\item[-]Traçabilité
\item[-]Intégrité des données.
\end{itemize}
\end{enumerate}
C'est un enjeu d'actualité qui 
