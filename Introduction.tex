\section{Introduction}
Le manque de qualité des données coûte environ 600 milliards de dollars à l’économie américaine chaque année. (Interaction between Record Matching and Data Repairing, Wenfei Fa et ali., 2011).\\
Ce constat alarmant montre la nécessité de s’intéresser au problème de la qualité des données, afin de le corriger en amont (faire la prévention, par le biais de M.D.M. par exemple), mais aussi en aval (de la correction, par le biais de “data cleaning”).
Le data cleaning est un sujet d’étude finalement assez récent, mais qui semble prometteur, puisque le marché du data cleaning est en hausse de 17\%, alors que le reste du marché de l’informatique est “seulement” en hausse de 7%.\\
Une brève définition de ce qu’est le data cleaning s’impose : l’objectif du data cleaning est de supprimer les erreurs et les incohérences d’un base de données afin d’améliorer la qualité des données.\\
Cependant, avant d’aborder plus avant le sujet du data cleaning à proprement parler, il est nécessaire d’aborder le sujet de la qualité de données. Comprendre l’origine et la diversité des problèmes de la qualité des données est nécessaire pour correctement aborder le sujet du data cleaning.\\
Nous verrons donc dans un premier temps quels peuvent être les différentes origines de la mauvaise qualité de données.\\
Les données fournies au système de nettoyage de données sont la plupart du temps d’origines diverses, et elle proviennent notamment souvent de bases de données différentes.\\
Ces origines diverses sont à l’origine de plusieurs problèmes concernant la qualité des données.\\